\documentclass[11pt,a4paper]{article} 
\usepackage{graphicx} 
\usepackage{amssymb, amsmath} 
\usepackage[margin=1cm]{geometry}
\usepackage[colorlinks=true,urlcolor=blue, linkcolor=blue]{hyperref}

\begin{document}
\title{Introduction to \LaTeX: Task Sheet}
\author{Chris Cooling\thanks{Material adapted from original by Alexis Belessiotis-Richards and Scott Melville}}
\maketitle
\pagestyle{empty}
\thispagestyle{empty}

\section{Creating a document}
\begin{itemize}
\item Create a new Overleaf project called ``Introduction to LaTeX'' - you will complete all exercises before Exercise \ref{sec:errors} in this project
\item Delete any sample text already in the project's .tex file
\item Create a new article document
\item Select a fontsize (11pt)
\item Select the type of page you want to write on (A4)
\end{itemize}

\section{Sections}
\begin{itemize}
\item Add a new chapter to your document (you will need to change the documentclass to ``report'' or ``book'')
\item Add a new section to your document
\item Add a new subsection to your document
\item Add a table of contents to your document
\end{itemize}

\section{Writing text}
\begin{itemize}
\item Add some text to your document
\item Make part of the text bold
\item Make part of the text italic
\item Underline part of the text
\item Change the size of some of your text
\end{itemize}

\section{Lists}
\begin{itemize}
\item Add a bullet-point list to your document
\item Add a numbered list to your document
\end{itemize}

\section{Equations}
\begin{itemize}
\item Add an equation to your document
\item Embed some mathmode content within some text
\item Start both of these pieces of mathmode content with simple content, then make them more complex
\item Try to include some maths relevant to your field - research how to include them online if necessary.
\end{itemize}

\section{Adding Figures}
\begin{itemize}
\item Add a figure to your document
\item Change the size of the figure
\item Change the vertical position of the figure
\item Centre the image horizontally on the page
\item Add a caption to the figure
\end{itemize}

If you receive an error, check the following common issues:

\begin{itemize}
\item You should be using the ``graphicx'' package
\item You should have uploaded your image file to Overleaf
\item Your file name shouldn't contain spaces
\item Your file should be a supported format (no ``svg'', ``tiff'' files)
\end{itemize}

\section{Adding Tables}
\begin{itemize}
\item Add a table to your document
\item Change the text alignment within each column
\item Add some horizontal and vertical lines
\item Add a caption to the table
\end{itemize}

\section{Referencing}
\begin{itemize}
\item Go back and add labels to your sections, subsections, equations, figures, and tables
\item Write some text and reference each of the labels you just added
\end{itemize}

\section{Bibliographies}
\begin{itemize}
\item Create a .bib file (you can export from a reference manager or create one from scratch)
\item Include the bibliography in your .tex file
\item Cite some of the bibliography entries in your document
\end{itemize}

If you receive an error, check the following common issues:

\begin{itemize}
\item You should be using the ``natbib'' package
\item You should have uploaded your .bib file to Overleaf
\item Your .bib file should not have a space in its filename
\item You should use the \textbackslash bibliographystyle command
\item You should use the \textbackslash  bibliography command. Make sure the name of the .bib file you specify is the same as the name of the file you uploaded
\end{itemize}

\section{Errors}\label{sec:errors}
\begin{itemize}
\item Make sure you've downloaded this file - the links won't work in the GitHub preview
\item Open \href{https://www.overleaf.com/read/gnzphcnqgfvf}{this LaTeX project}
\item Make a copy of the project (click the "Menu" button, then "Copy Project")
\item Correct the errors and compile the pdf
\item If you get stuck or want to compare your answer, the corrected version can be found \href{https://www.overleaf.com/read/zsmyvxvdcvfr}{here}
\end{itemize}

\end{document}
